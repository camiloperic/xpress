\part{Considerações Finais}
Este trabalho é um Trabalho de Graduação para o curso de Sistemas de Informação da Escola Superior de Engenharia e Gestão. O experimento desenvolvido neste trabalho cumpre o objetivo proposto de desenvolver um protótipo de jogo educativo de expressões algébricas. O protótipo é capaz de gerar e avaliar expressões com somas e subtrações, além de permitir a solução das expressões pelo usuário. O protótipo também conta com expressões com as quatro operações em sua base de dados.

Existem diversas possibilidades de continuação deste trabalho, essas possibilidades surgem tanto de aspectos incompletos assim como aspectos que podem ser aperfeiçoados. Como apontado no texto sobre as motivações envolvidas neste trabalho existe um grande potencial na utilização de informações que podem ser geradas pelo sistema. O algoritmo de avaliação de soluções além de ter um desempenho fraco para árvores grandes este funciona apenas para somas e subtrações.

 	As informações relevantes que podem ser armazenadas devem estar estruturadas para que a possibiliade de análise dos dados seja maior. Tais informações podem incluir erros mais comuns, quantidade de exercícios feitos, aproveitamento para um determninado tópico, entre muitos outros. Tais informações geradas ainda podem retroalimentar o sistema, por exemplo, o jogo pode escolher o  nível de dificuldade do problema para determinado aluno ou escolher determinados tipos de problemas que o aluno tenha menor aproveitamento, entre outras possibilidades. %
Uma possibilidade de continuação deste trabalho é adicionar um banco de dados ao sistema e estudar que tipos de análise podem ser feitas sobre as informações geradas e o que se pode aprender com estas possíveis análises.

A melhoria do desempenho do algoritmo de avaliação de soluções de expressões também pode ser objeto de um trabalho que se extende deste. Além do desempenho do algoritmo, o escopo das expressões que podem ser avaliadas pode ser expandido para incluir multiplicações e divisões.

Outro problema interessante que surgiu durante o trabalho mas não foi resolvido foi como criar expressões algébricas com somas, subtrações, multiplicações e divisões cujos números presentes sejam sempre inteiros.

Outro desdobramento possível para este trabalho é incluir novas “operações” como potenciação e radiciação, desta forma o programa pode abordar as propriedades de potenciação e radiciação por exemplo.

O código-fonte do jogo está disponível para qualquer interessado utilizando o serviço de repositórios GitHub em http://github.com/camiloperic/xpress.
