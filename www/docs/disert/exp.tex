\part{Experimento}
O experimento desenvolvido neste trabalho consiste na criação de um jogo educacional utilizando a tecnologia HTML5. Como tecnologias complementares foram utilizadas: Queue.js, Highcharts e PhoneGap. O conteúdo educacional escolhido para ser trabalhado no jogo foi expressões algébricas. O jogo consiste em apresentar uma expressão ao jogador e pedir para que ele selecione e resolva cada uma das operações até chegar ao resultado final.

\chapter{Modelo de dados}
Por mais que o experimento não utilize uma aplicação de banco de dados existe um modelo de dados intrínseco expresso no diagrama a seguir.

\begin{figure}[H]
	\caption{\label{gram_cls}Modelo de Dados}
	\begin{center}
	    \includegraphics[scale=0.9]{datamodel.png}
	\end{center}
	\legend{Fonte: Produzido pelo autor}
\end{figure}

O diagrama representa as expressões e suas respectivas soluções. Como a estrutura de dados escolhida para representar as expressões foi a árvore, as expressões podem ser armazenadas da mesma forma. Cada nó (Node) aponta para o seu nó pai e seus nós filhos.

As soluções (Solution) para uma determinada expressão são armazenadas na forma de uma lista ordenada de transformações (Transformation). A lista ordenada de transformações é representada pela entidade SolutionTransformation.

\chapter{Visão geral do fluxo}
O fluxo tem início quando o usuário acessa o jogo – como página de internet ou aplicativo para dispositivo móvel. O jogo inicializa e apresenta o botão “Play” para que o jogador comece os testes. Quando iniciados os testes o jogo escolhe uma expressão e a apresenta ao usuário para que ele selecione uma operação para resolver.

Assim que uma operação é selecionada o programa avalia se esta pode ser resolvida e apresenta a pontuação. Caso a operação selecionada não possa ser resolvida o programa se mantém no estado de seleção, no caso contrário o programa pede a solução para a operação selecionada. Depois que o usuário passa a solução o jogo apresenta a pontuação. Se a solução passada estiver incorreta o jogo permanece no estado de resolução, se a solução estiver correta a aplicação apresenta a expressão resultante.

Enquanto houver operações não resolvidas o fluxo se repete desde a seleção de operação. Quando não houver mais operações a serem resolvidas o programa apresenta os botões “Next...” e “Exit” para o jogador fazer o próximo teste ou sair do jogo respectivamente. Se o jogador for para o próximo teste o fluxo volta para o passo de escolha de expressão, no caso do jogador sair o programa apresenta a tela final.

\section{Inicialização da aplicação}
A função start é responsável pela inicialização da aplicação, recebe como parâmetro o identificador do elemento HTML que irá conter a tela do jogo e insere os principais elementos HTML do jogo, inclusive a tela inicial para que o usuário possa iniciar o jogo.

\section{Escolha da expressão}
A escolha da expressão é feita por meio da função newXp. Inicialmente a rotina seleciona uma expressão, escolhendo uma das disponibilizadas pelo método startDB (executado no início dos testes) ou gerando uma com a função makeExp. Em seguida a expressão selecionada é passada para o contexto do jogo junto de suas soluções.

A criação de novas expressões leva em conta duas variáveis: número de operações e um conjunto de possíveis operações a serem utilizadas. As expressões são criadas de forma aleatória. A probabilidade para cada operação ser sorteada é 1 em $n$, onde $n$ é o tamanho do conjunto de operações passado. 	Não existe restrição de unicidade para o conjunto de operações que são passadas para a função makeExp possibilitando alterar a probabilidade de que uma determinada operação seja escolhida.

Caso a expressão escolhida tenha sido gerada a fase de passar a expressão para o contexto do jogo envolve a avaliação da expressão para conhecer todas as soluções possíveis para a expressão. A avaliação da expressão é feita utilizando um algoritmo iterativo implementado pela função evaluateTreeIt.

\section{Apresentação da expressão}
A expressão é apresentada utilizando a função appendXp que recebe como parâmetro a expressão e a imprime na tela utilizando a função htmlfy, uma função recursiva que retorna os elementos HTML que representam visualmente a expressão.

\section{Selecão da operação}
A seleção da operação é acionada por um clique simples com o botão esquerdo do mouse ou toque (no caso de dispositivos móveis) sobre a operação. A função opClick é responsável por tratar tais eventos recebendo como parâmetro o identificador da operação. Internamente a função opClick faz uso da função select que é responsável por avaliar se a operação pode ou não ser resolvida.

\section{Pontuação}
Sempre que houver uma seleção ou tentativa de resolver uma operação o jogo avalia se a ação está correta e então apresenta a pontuação por meio da função spanWarn que recebe como parâmetros o tipo de pontuação, perda ou ganho, e o respectivo valor.

Quando o jogador acerta uma seleção ou resolução na primeira tentavia ele recebe um ponto inteiro. Para cada erro o ponto é divido ao meio, assim se o acerto ocorrer na segunda tentativa o jogador recebe meio ponto. A seleção da última operação não é pontuada por ser uma opção única.

\section{Pedido de solução}
Sempre que uma operação for selecionada corretamente o jogo pede ao usuário a solução utilizando a função askForSolution que apresenta ao usuário a operação isolada seguida de uma caixa de texto, para que o usuário insira a solução daquela operação. Os elementos HTML apresentados contém aqueles retornados pela função htmlfy.

\section{Passagem da solução}
Depois que o usuário inserir a solução na caixa de texto e apertar a tecla ENTER ou tirar o foco da caixa de texto o programa faz uma avaliação da solução passada pela função solSubmit. Caso a solução esteja correta a caixa de texto é transformada em texto simples. Se houver mais de uma possível solução apenas aquela que foi resolvida permanece na tela, as outras são eliminadas da tela.

\section{Exemplo}
O exemplo a seguir exemplifica como o fluxo do jogo é percebido pelo usuário através da interface. Após inicializar a aplicação o usuário vê a tela inicial e o botão “Play”. Após apertar o botão “Play” o jogo apresenta uma expressão para o jogador resolver.

\begin{figure}[H]
	\caption{\label{xp_1} O sistema apresenta a expressão a ser resolvida}
	\begin{center}
	    \includegraphics[scale=1]{xp_4_1.png}
	\end{center}
	\legend{Fonte: Produzido pelo autor}
\end{figure}

O jogador seleciona uma operação que não pode ser resolvida e perde metade da pontuação da seleção.

\begin{figure}[H]
	\caption{\label{miss_0_5_1}Jogador perde metade da pontuação da seleção}
	\begin{center}
	    \includegraphics[scale=1]{miss_0_5.png}
	\end{center}
	\legend{Fonte: Produzido pelo autor}
\end{figure}

Em seguida o usuário seleciona uma operação que pode ser resolvida e ganha a pontuação da seleção.

\begin{figure}[H]
	\caption{\label{score_0_5_1}Jogador ganha a pontuação da seleção}
	\begin{center}
	    \includegraphics[scale=1]{score_0_5.png}
	\end{center}
	\legend{Fonte: Produzido pelo autor}
\end{figure}

O programa então apresenta a operação isolada e a caixa onde o usuário insere a solução. Este passo será omitido no resto do exemplo.

\begin{figure}[H]
	\caption{\label{xp_2}Programa pede a solução da operação}
	\begin{center}
	    \includegraphics[scale=1]{xp_4_2_asksol_1.png}
	\end{center}
% 	\legend{Fonte: Produzido pelo autor}
\end{figure}

O jogador erra a solução e perde metade da pontuação da resolução.

\begin{figure}[H]
	\caption{\label{xp_3}O usuário insere um valor errado como solução}
	\begin{center}
	    \includegraphics[scale=1]{xp_4_3_wrongans_1.png}
	\end{center}
	\legend{Fonte: Produzido pelo autor}
\end{figure}

\begin{figure}[H]
	\caption{\label{miss_0_5_2}O jogador perde metade dos pontos da solução}
	\begin{center}
	    \includegraphics[scale=1]{miss_0_5.png}
	\end{center}
	\legend{Fonte: Produzido pelo autor}
\end{figure}

Em seguida o valor é corrigido e o usuário ganha a pontuação da solução.

\begin{figure}[H]
	\caption{\label{xp_4}O jogador insere a solução correta}
	\begin{center}
	    \includegraphics[scale=1]{xp_4_4_rightans_1.png}
	\end{center}
	\legend{Fonte: Produzido pelo autor}
\end{figure}

\begin{figure}[H]
	\caption{\label{score_0_5_2}O sistema pontua o jogador pela solução}
	\begin{center}
	    \includegraphics[scale=1]{score_0_5.png}
	\end{center}
	\legend{Fonte: Produzido pelo autor}
\end{figure}

O campo de texto é transformado em texto simples e a expressão resultante é apresentada ao usuário. O fluxo seleção/resolução se repete.

\begin{figure}[H]
	\caption{\label{xp_5}O sistema mostra a expressão resultante}
	\begin{center}
	    \includegraphics[scale=1]{xp_4_5.png}
	\end{center}
	\legend{Fonte: Produzido pelo autor}
\end{figure}

\begin{figure}[H]
	\caption{\label{score_1_1}O jogador recebe 1 ponto pelo acerto da seleção}
	\begin{center}
	    \includegraphics[scale=1]{score_1.png}
	\end{center}
	\legend{Fonte: Produzido pelo autor}
\end{figure}

A divisão pode ser resolvida de duas maneiras e o jogo apresenta as duas possibilidades de solução.

\begin{figure}[H]
	\caption{\label{xp_7}O jogo aparesenta duas possibilidades de resolução para a divisão}
	\begin{center}
	    \includegraphics[scale=1]{xp_4_7_rightans_2.png}
	\end{center}
	\legend{Fonte: Produzido pelo autor}
\end{figure}

\begin{figure}[H]
	\caption{\label{score_1_2}O jogador recebe 1 ponto pelo acerto da solução}
	\begin{center}
	    \includegraphics[scale=1]{score_1.png}
	\end{center}
	\legend{Fonte: Produzido pelo autor}
\end{figure}

O fluxo seleção/resolução se repete.

\begin{figure}[H]
	\caption{\label{xp_8}A nova expressão é apresentada ao usuário}
	\begin{center}
	    \includegraphics[scale=1]{xp_4_8.png}
	\end{center}
	\legend{Fonte: Produzido pelo autor}
\end{figure}

A seleção da última operação não é pontuada.

\begin{figure}[H]
	\caption{\label{score_0_1}A seleção da última operação não é pontuada}
	\begin{center}
	    \includegraphics[scale=1]{score_0.png}
	\end{center}
	\legend{Fonte: Produzido pelo autor}
\end{figure}

\begin{figure}[H]
	\caption{\label{xp_10}O jogador insere a última solução que está correta}
	\begin{center}
	    \includegraphics[scale=1]{xp_4_10_rightans_3.png}
	\end{center}
	\legend{Fonte: Produzido pelo autor}
\end{figure}

\begin{figure}[H]
	\caption{\label{score_1_3}O jogador recebe a pontuação referente a solução}
	\begin{center}
	    \includegraphics[scale=1]{score_1.png}
	\end{center}
	\legend{Fonte: Produzido pelo autor}
\end{figure}

	Não existem mais operações para serem resolvidas e as opções de continuar ou sair são apresentadas ao usuário.
	
\begin{figure}[H]
	\caption{\label{xp_11}A tela final é apresentada}
	\begin{center}
	    \includegraphics[scale=1]{xp_4_11.png}
	\end{center}
	\legend{Fonte: Produzido pelo autor}
\end{figure}

\chapter{Criação de expressões}

As expressões algébricas utilizadas são linguagens formais pois são um subconjunto de todas as palavras existentes para um determinado alfabeto. Para tanto existe pelo menos uma gramática capaz de gerar a linguagem formal envolvida.

\section{A linguagem formal envolvida no jogo}
Vamos partir do conjunto de símbolos terminais que deve conter:

\begin{alineas}
\item as operações de soma, subtração, multiplicação e divisão;
\item um conunto de números dado por um intervalo fechado sobre $\mathbb{Z}$; 
\item parênteses.
\end{alineas}

Logo temos $T = \{+, -, *, /, -2, -1, 0, 1, 2, (, )\}$. Escolhemos um pequeno intervalo de inteiros neste caso para manter o conjunto pequeno.

	As expressões que fazem da linguagem formal envolvida são do tipo infixa, ou seja, a operação é posicionada no meio de dois elementos que podem ser outras uma outra operação ou um inteiro. Esta escolha se dá por ser o formato que os alunos estão acostumados a ver.
	
\subsection{Prefixo, infixo, pósfixo}
Podemos representar uma operação matemática utilizando os três seguintes formatos: prefixo, infixo e pósfixo. No primeiro formato a operação é posicionada em primeiro, antes dos dois operandos. O formato pósfixo é simétrico ao formato anterior, ou seja, a operação é posicionada em último depois dos operandos. Uma vantagem da utilização dos formatos anteriores é a não necessidade de utilizar parênteses para expressar precedência entre as operações. Já no formato infixo a operação é posicionada entre os operandos e necessita utilizar parênteses para expressar precedência entre as operações.

\subsection{Parênteses}
A necessidade de utilizar parênteses para expressar precedência entre operações na forma infixa não ocorre para todas as operações na expressão. Os casos são os seguintes: 

\begin{alineas}
\item quando uma subtração tem como operando direito outra operação que seja uma soma ou subtração;
\item quando o operando de uma multiplicação ou divisão for uma operação de soma ou subtração;
\item quando um número negativo é operando direito de uma operação.
\end{alineas}

\subsection{Gramática referência}
Tendo em conta os pontos apresentados anteriormente é possível criar uma gramática que seja capaz de gerar as expressões que são utilizadas no jogo. Algumas simplificações serão feitas a seguir nas produções: os terminais $o$ e $i$ representam os seguintes conjuntos de terminais respecitvamente as operações envolvidas e um intervalo fechado sobre $\mathbb{Z}$.

	Com a simplificação anterior não existe necessidade de criar uma regra de produção para cada operação e para cada inteiro. Subentende-se que estes terminais, que representam conjuntos, poderiam se desdrobrar em um símbolo não-terminal que produz cada um dos elementos do conjunto que representam sozinho.
	
	A gramática pensada como referência para a solução adotada na criação das expressões é a seguinte:
	
$G: (N, T, P, S)$, onde:

$N = \{S,E\}$,

$T = \{o,i,(,)\}$,

$P$ é o conjunto das seguintes produções:

$S \to (EoE)$

$E \to (EoE) | (i)$,

$\sigma = S$.

Note que a linguagem possui parênteses para todas as operações e inteiros. A linguagem foi pensada assim pois a forma ecolhida para representar suas palavras não foi um texto simples mas sim uma árvore que por natureza já expressa as precedências entre as operações além da necessidade dos parênteses para alguns números inteiros negativos. A lógica de apresentação de parênteses está encapsulada na função htmlfy.

	De acordo com a Classificação de Chomsky\footnote{Apresentada no tópico 1.1 Classificação de Chomsky} tal gramática é Livre de Contexto.
	
\section{Algoritmo para criação de expressões}
No código Javascript contido no arquivo xpress.js temos a função makeExp responsável pela criação das expressões. A função recebe como parâmetros o número de operações que a expressão deve conter e um conjunto contendo quais operações deversão fazer parte da expressão. A lógica da função makExp está encapsulada em duas funções, makeOps e fillWithInts, onde a primeira gera uma árvore de operações apenas e a segunda preenche a árvore com inteiros, lembrando que a árvore utilizada para representar as expressões é do tipo binária já que cada operação possui dois operandos.

\subsection{makeOps}
A função makeOps tem os mesmos parâmetros da função makeExp. Caso as operações não sejam passadas a função utiliza todas as operações (soma, subtração, multiplicação e divisão). A rotina começa com a criação de: uma lista de possíveis operações pai iniciada vazia; uma referência para a raíz da árvore de operações que começa com o valor nulo.

	O passo seguinte é uma iteração sobre o número de operações a serem criadas. Uma operação daquelas passadas como parâmetro é sorteada e passada como parâmetro para a função criadora dos nós que retorna um nó com aquela determinada operação. Se a referência a raíz feita fora da iteração ainda possuir o valor nulo a referência apontará então para o nó criado. No caso contrário o programa sorteia um possível pai da lista, remove da mesma, definindo este como pai do nó criado. 
	
	No final da iteração o programa adiciona novas entradas na lista de possíveis pais. Estas entradas representam a ideia de que o ultimo nó criado pode ser pai a esquerda e pai a direita. Quando não existem mais iterações para acontecer a função retorna o nó referenciado como raíz.
	
\subsection{fillWithInts}
No caso de uma árvore incompleta, ou seja, existe pelo menos um operando de uma operação que não foi definido a função fillWithInts pode torná-la completa. A função recebe como parâmetro a raíz da árvore a ser preenchida com inteiros. A função faz uma busca em profundida na árvore para encontrar todas as operações que possuem operandos não definidos e então os define.

	A busca em profundidade é feita utilizando uma pilha de nós a serem analisados. A pilha começa com a raíz que foi passada como parâmetro. Enquanto houver nós na pilha o programa tira o último nó da pilha testando se seus operandos, esquerdo e direito, são nós ou se estão vazios. No caso de o operando ser uma operação o programa o empilha para ser futuramente analisado, se não a rotina cria um nó com um número inteiro e o define como o operando que falta.
	
\subsection{Restrição}
Embora o algoritmo criado seja capaz de criar expressões com todas as operações (soma, subtração, multiplicação e divisão) no experimento este é utilizado somente para criar expressões com somas e subtrações, as duas com a mesma probabilidade.

	A restrição foi feita para limitar o escopo deste trabalho pois como requerimento todos os números envolvidos nas expressões, inclusive aqueles que são a solução de uma operação, devem ser inteiros. Neste caso se as expressões possuem divisão o algoritmo na expressão fillWithInts deveria levar tal fato em conta e garantir que os resultados das divisões sejam números inteiros.

\chapter{Análise de soluções}
As expressões algébricas envolvidas no experimento podem ter mais de uma forma de solução e para que o jogo possa avaliar corretamente a resolução da expressão feita pelo usuário é necessário fazer a avaliação da expressão a fim de encontrar todas as ordens de resolução possível.

\section{Ambiguidade}
A possibilidade de mais de uma solução para uma mesma expressão pode ser explicada por meio da propriedade associativa das operações\footnote{Explicada em 2. Operações no Conjunto dos Números Inteiros}. Por exemplo, $3+4+5$ é uma expressão ambigua pois poderia ser resolvida em duas ordens:

$3+(4+5)=$ 

$3+9=$

$12$

, ou,

$(3+4)+5=$

$7+5=$

$12$
     
As duas expressões iniciais são equivalentes a expressão apresentada $3+(4+5)=(3+4)+5=3+4+5$. A ausência de parênteses na última expressão mostra que não há precedência obrigatória entre as duas operações de soma, ou seja, que ela pode ser resolvida em diferentes ordens. Já a expressão $5-(4+3)$ não é ambigua em relação à ordem de resolução e possui apenas uma forma de ser resolvida. Isto acontece pois a propriedade associativa não é valida para as operações envolvidas ($+,-$). Os parênteses expressam uma ordem de precedência existente e desta forma são obrigatórios, se retirados o valor da expressão será alterado.

\subsection{Rotação das árvores}
Para as expressões envolvidas no jogo a ordem dos símbolos não pode ser alterada, o jogo não lida com a propriedade comutativa na análise de soluções. A ordem dos símbolos das expressões é dada pela ordem transversal dos nós\footnote{Explicada em 3. Árvores binárias ordenadas e rotações}. Os parênteses não são nós da árvore que representa a expressão e estão implicitos na estrutura da seguinte forma, se uma operação de multiplicação tem como um de seus operandos o resultado de uma operação de soma, a operação de soma envolvida deve estar dentro de parênteses para expressar a precedência existente entre as operações.

Como apresentado no capítulo 3, a operação de rotação sobre uma árvore binária não altera a ordem transversal dos nós. No caso de uma expressão sem precedência entre as operações podemos rotacionar a árvore, que representa a expressão, indeterminadamente alterando sua estrutura mas não o significado (operadores e operandos em sua ordem original). Porém, no caso de expressões com parênteses necessáriosa árvore não pode ser rotacionada livremente e as restrições em relação à estas rotações é dada pela lógica que explica a necessidade dos parênteses.

Observando os exemplos de expressões dados anteriormente ($3+4+5$ e $5-(4+3)$) podemos entender melhor as rotações e suas restrições. No caso da primeira expressão, devido a propriedade associativa da soma sobre o conjunto dos números inteiros, uma soma pode ser o operando da outra e vice-versa. Na expressão seguinte a soma é operando da subtração apenas (precedência expressa na árvore pela relação de pai e filho existente entre a subtração e a soma).

\subsubsection{Rotação horária}
	A rotação neste sentido é sempre possível para as operações que possuem como filho esquerdo outra operação, já que a única restrição a ser considerada para saber se a rotação pode ser feita se refere apenas ao filho direito do nó a ser rotacionado.
	
\subsubsection{Rotação anti-horária}
	A rotação anti-horária não pode ser feita para toda operação que tem como filho direito outra operação, a operação pai em questão não pode ser uma subtração (caso da segunda expressão utilizada como exemplo anteriormente, $5-(4+3)$).
	
\subsection{Números de Catalan}
Para um expressão que contenha apenas operações de multiplicação o número de ordens possíveis de solução é dado pelo $n$-ésimo número de Catalan, onde $n$ é o número de operações.\footnote{Característica explicada em 4. Números de Catalan} Esta característica é consequência da propriedade associativa da multiplicação sobre o conunto dos números inteiros e pode ser extendida a uma expressão só com soma já que a soma também é associativa em $\mathbb{Z}$

Como as expressões a serem avaliadas envolvem operações de soma e subtração nem todas as rotações são possíveis, sendo assim para conhecer o número de soluções não é possível apenas utilizar o $n$-ésimo número de Catalan. Se particionarmos a árvore exatamente onde existem as restrições podemos conhecer o número de soluções para cara partição (subárvore) e então combiná-los para encontrar o número total de soluções para uma determinada expressão.
	
\section{Algoritmo para análise de soluções}
A criação de expressões é um ponto importante no experimento desenvolvido neste tabalho pois evita a necessidade da criação manual de expressões e cria um potencial repositório infinito de expressões. Para tanto é necessário que a aplicação seja capaz de avaliar a expressão criada para encontrar todas as possíveis soluções.

  Para facitilar a explicação do algoritmo um pseudocódigo e um mapa de execução do algoritmo estão disponíveis como ANEXO A e ANEXO B respectivamente. O pseudocódigo explica a lógica geral envolvida e não entra em detalhes mais específicos de funcionamento do algoritmo. O mapa de execução é um exemplo de uma instancia de execução do algoritmo para uma árvore com 3 nós internos.

O ponto de partida do algoritmo é uma expressão (lembrando que as expressões são representadas por árvores binárias), esta é a expressão avaliada pelo algoritmo. O primeiro laço que ocorre no algoritmo é a iteração de uma fila de soluções, que inicialmente possui apenas a expressão passada inicialmente ao algoritmo. No pseudocódigo este laço está na linha 5 - controlado na linha 7 quando a primeira solução saí da fila iterada -, a fila de soluções é representada por "Soluções a avaliar" e a expressão inicial por "Árvore" . No mapa de soluções a fila pode ser vista como a primeira linha da tabela que contém as árvores que representam as soluções (S0, S1, S2, S3, S4), S0 é a solução inicial (ou expressão inicial) e a iteração ocorre da esquerda para a direita.

O laço seguinte , que está dentro do anterior, itera uma lista que contém as operações contidas na solução corrente (em relação ao laço no qual este está contido) S na ordem inversa da ordem que resulta da passagem de um algoritmo de busca em largura sobre S. Esta lista está representada no pseudocódigo por "Nós a avaliar" cuja iteração está na linha 12. No mapa de soluções estas iterações estão representadas nas colunas cinzas ("Nós") e a ordem da iteração ocorre de cima para baixo.

No começo do laço sobre as operações uma lista é criada a partir do produto cartesiano das soluções dos dois nós filhos desta operação (armazenadas em um mapa de soluções por nó). Esta lista é referida como "Pré-soluções" no pseudocódigo e como colunas azuis no mapa de execução. A avaliação do nó corrente N (da iteração sobre os nós) ocorre uma vez para cada elemento da lista de pré-soluções e este é o terceiro laço - encadeado no anterior. A iteração da lista de pré-soluções está na linha 16 do pseudocódigo e a ordem das iterações no mapa de soluções é de cima para baixo.

Para cada pré-solução iterada a árvore inicial é alterada de acordo e então N é testado para possíveis rotações horárias e anti-horárias, primeiro no sentido horário e depois no sentido anti-horário. Estas sequências de testes são os dois próximos laços (um para cada sentido de rotação), laços que estão na mesma linha de execução (dentro da iteração das pré-soluções). No pseudocódigo estes laços podem ser observados nas linhas 24 e 36. Já no mapa de execução cada rotação está representada pelas células amarelas e vermelhas (possíveis rotações horárias e anti-horárias respectivamente).

Caso N seja raíz cada possível rotação, horária ou anti-horária, será testata como uma possível nova solução. O teste de novas soluções é feito por meio da comparação de uma chave que representa cada possível árvore como uma determinada ordem dos nós. No pseudocódigo o mapa "Soluções" contém o conjunto de chaves para as soluções já encontradas e os testes para novas soluções estão nas linhas 28 e 40. Cada possível rotação contém a subárvore resultante no mapa de execução (e.g. a primeira rotação possível - anti-horária - ocorre no nó b da solução S0 e seu resultado é dado pela subárvore da solução S4 que começa em c).

% 	O algoritmo utilizado para a análise de soluções analisa a árvore que representa a expressão dos nós mais distantes da raíz até os mais próximos. Desta forma cada nó pai conhece todas as transformações possíveis para seus nós filhos. A partir do produto cartesiano de todas as transformações possíveis o nó é testado para rotações sobre todas as transformações contidas no produto cartesiano de soluções dos filhos.
	
% 	Quando o nó a ser avaliado for raíz o algoritmo testa para conhecer se as soluções avaliadas já existem em um conjunto final de soluções ou não, caso não o algoritmo adiciona tal soluções a este conjunto.
	
% 	Caso uma nova solução seja encontrada esta é colocada em uma fila para que seja analisada em seguida, assim como a árvore inicial foi. O pseudo-código a seguir explica em mais detalhes o funcionamento do algoritmo para uma dereminada árvore:

\subsection{Análise de complexidade e desempenho}
Para a análise de complexidade do algoritmo vamos partir das instruções que são executadas menos vezes e chegar até as instruções que são executadas o maior número de vezes. As instruções que estão fora dos laços são executadas uma vez apenas e a constante $g$ representa a quantidade destas instruções. $C$ representa o número total de instruções executadas para uma árvore com $n$ operações:

$C = g + ...$

Como esclarecido no tópico anterior a estrutura do algoritmo contém cinco laços principais. Seja $n$ o número de operações envolvidas na expressão ($n \in \mathbb{Z}*$) o primeiro laço é executado $s$ vezes, onde $s \in \mathbb{Z}$ e $1 \leqslant s \leqslant C(n)$ ($C(n)$ representa o $n$-ésimo número de Catalan). Sendo $h$ o número de instruções executadas dentro deste primeiro laço apenas $hc$ representa o total de instruções executadas dentro do laço em todas as suas iterações.

$C = g + hc + ...$

O laço imediamente seguinte é a iteração sobre as operações envolvidas na expressão e é executado $n$ vezes para cada iteração do laço anterior, quantidade que pode ser expressa como $cn$. Seja $i$ uma constante que representa o número de instruções que estão imediatamente abaixo do laço sobre as operações a quantidade dessas instruções executadas durante uma availação é dada por $icn$.

$C = g + hc + icn + ...$

Em seguida as pré-soluções são analisadas. O número de pré-soluções $p$ varia de acordo com a subárvore testada sendo que $1 \leqslant p \leqslant C(n-1)$. A constante $j$ representa as instruções diretamente ligadas ao laço que itera as pré-soluções e $jcnp$ o quantidade total de execução destas instruções.

$C = g + hc + icn + jcnp + ...$

Os dois laços seguintes, testes de rotações, tem como o número total de iterações (somadas) $r$, onde $r \in \mathbb{Z}$ e $0 \leqslant r \leqslant n-1$. Como os dois laços têm o mesmo número de operações, $k$, a quantidade de instruções executadas nestes laços durante todo o algoritmo é dada por $kcnpr$. Sendo assim temos como consumo total:

$C = g + hc + icn + jcnp + kcnpr$

O algoritmo tem como característica então o consumo de tempo polinomial expresso pela fórmula apresentada anteriormente. Para testar o desempenho do algoritmo de avaliação de soluções a função evaluateTreeItTest foi criada. A função recebe quatro parâmetros de teste. Os dois primeiros são os limites de um intervalo de números inteiro que representam os tamanhos de árvores a serem testadas. O parâmetro seguinte se refere ao número de testes por tamanho de árvore. O ultimo parâmetro é opcional, uma lista de possíveis operações a serem utilizadas na criação das expressões que serão testadas.

	Nos gráficos em anexo (ANEXO C e ANEXO D) podemos observar os resultados dos testes. Os teste foram feitos 10 vezes para árvores de 3 a 9 nós, onde no primeiro teste somente operações de soma foram utilizadas diferente so segundo que também utiliza subtrações.

A diferença nos tempos das séries sem e com soma se devem ao fato de que expressões com subtração tem limitações de rotação. O primeiro gráfico em escala linear (ANEXO C) mostra o comportamento polinomial da expressão. No gráfico cuja a escala é logarítmica (ANEXO D) podemos observar com maior clareza como os tempos nas expressões que envolvem subtração variam em um intervalo cujo tempo máximo é o tempo para as árvores que não envolvem subtrações. Isto se deve ao caráter aleatório do algoritmo de geração de expressões. Uma expressão só com somas pode ser gerada mesmo no teste que envolve também subtrações.
