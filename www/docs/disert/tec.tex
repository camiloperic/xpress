\part{Tecnologias}
% ----------------------------------------------------------

% ---
% primeiro capitulo de Resultados
% ---
\chapter{HTML5}
De acordo com o texto de introdução de W3SCHOOLS HTML5 é o padrão HTML mais recente. A versão anterior (4.01) foi lançada em 1999 e como a internet mudou significantemente desde então a versão 5 vem para substituir a versão 4.01, além do XHTML e o HTML DOM Level 2. A nova versão provê de animações a gráficos, musicas a filmes além de possibilitar aplicações web complexas. O padrão HTML5 é multiplataforma e é feito para funcionar em PCs, Tablets, Smartphones e Smart TVs.
	O versão 5 do padrão HTML é uma cooperação entre o World Wide Web Consortium (W3C) e o Web Hypertext Application Tecnology Working Group (WHATWG). Ambas as organizações trabalhavam em diferentes aspectos de aplicações web e em 2006 resolveram se juntar para criar a nova versão do HTML. Algumas diretrizes foram estabelecidas para o futuro empreendimento \cite{htmlInt}:

\begin{alineas}

\item Novas funcionalidades devem estar baseadas em HTML, CSS, DOM e JavaScript;

\item A necessidade de plugins externos deve ser reduzida;

\item Lidar com erros deve ser mais fácil que nas versões anteriores;

\item “Scripting” deve ser substituido por mais marcações;

\item O padrão deve ser independente de dispositivo;

\item O publico deve ter visibilidade do processo de desenvolvimento

\end{alineas}

Entre as novas “features” disponibilizadas na nova versão temos:

\begin{alineas}

\item O elemento canvas para desenho em 2D;

\item Os elementos video e audio para execução de mídias;

\item Suporte para armazenamento local

\end{alineas}

\section{O surgimento do HTML}
A World Wide Web teve inicio no CERN, Laboratório para Física de Particulas em Geneva, Suiça. O conceito do HTML surgiu enquanto Tim Berners-Lee trabalhava em uma sessão de serviços computacionais do CERN. Tim teve a ideia de permitir que pesquisadores em diferentes lugares do mundo pudessem organizar e juntar informações remotamente, já que as pesquisas em Física de Particulas envolve com frequência diversos institutos de varios lugares do mundo. A questão não era apenas disponibilizar um grande número de pesquisas mas permitir que ligações entre os documentos fossem estabelecidas. Antes de trabalhar no CERN Tim havia trabalhado com produção de documentos e processamento de texto. Tim pensou que a solução seria desenvolvida através de uma forma de hipertexto.

O protótipo de navegador web estava pronto em 1990 para o computador NeXT e foi desenvolvido por Tim Berners-Lee. O surgimento da web no começo dos anos 90 está relacionado aos desenvolvimentos em tecnologias de comunicação durante o período assim como o hipertexto ganhava espaço e comçava a ser usado em computadores. O sistema de nomes de domínios (DNS) também foi importante no surgimento da Web pois facilitava o acesso as máquinas específicas.

Tim Berners-Lee via a viabilidade dos links globais de hipertexto. Uma consideração importante feita por Tim foi a importância de que tal hipertexto deveria funcionar nos mais diversos computadores que estavam ligados à Internet. Tim desenvolve um software e um protocolo (HTTP) que eram capazes de recuperar documentos de texto através de links de hipertexto para demonstrar sua forma de publicar documentos. O texto que era utilizado pelo protocolo HTTP (HyperText Transfer Protocol) foi nomeado de HTML que significa linguagem de marcação de hipertexto (HyperText Mark-up Language).

A linguagem HTML criada por Tim Berners-Lee se baseou em um método de marcação de textos que organizava o texto em unidades estruturais como parágrafos, cabeçalhos entre outras, o SGML que era um método acordado internacionalmente. SGML significa linguagem padrão generalizada de marcação (SGML). Tal embasamento foi um fator que colaborou para o sucesso da ideia de Tim \cite{htmlHist}. A World Wide Web teve inicio no CERN, Laboratório para Física de Particulas em Geneva, Suiça. O conceito do HTML surgiu enquanto Tim Berners-Lee trabalhava em uma sessão de serviços computacionais do CERN. Tim teve a ideia de permitir que pesquisadores em diferentes lugares do mundo pudessem organizar e juntar informações remotamente, já que as pesquisas em Física de Particulas envolve com frequência diversos institutos de varios lugares do mundo. A questão não era apenas disponibilizar um grande número de pesquisas mas permitir que ligações entre os documentos fossem estabelecidas. Antes de trabalhar no CERN Tim havia trabalhado com produção de documentos e processamento de texto. Tim pensou que a solução seria desenvolvida através de uma forma de hipertexto.

O protótipo de navegador web estava pronto em 1990 para o computador NeXT e foi desenvolvido por Tim Berners-Lee. O surgimento da web no começo dos anos 90 está relacionado aos desenvolvimentos em tecnologias de comunicação durante o período assim como o hipertexto ganhava espaço e comçava a ser usado em computadores. O sistema de nomes de domínios (DNS) também foi importante no surgimento da Web pois facilitava o acesso as máquinas específicas.

Tim Berners-Lee via a viabilidade dos links globais de hipertexto. Uma consideração importante feita por Tim foi a importância de que tal hipertexto deveria funcionar nos mais diversos computadores que estavam ligados à Internet. Tim desenvolve um software e um protocolo (HTTP) que eram capazes de recuperar documentos de texto através de links de hipertexto para demonstrar sua forma de publicar documentos. O texto que era utilizado pelo protocolo HTTP (HyperText Transfer Protocol) foi nomeado de HTML que significa linguagem de marcação de hipertexto (HyperText Mark-up Language).

A linguagem HTML criada por Tim Berners-Lee se baseou em um método de marcação de textos que organizava o texto em unidades estruturais como parágrafos, cabeçalhos entre outras, o SGML que era um método acordado internacionalmente. SGML significa linguagem padrão generalizada de marcação (SGML). Tal embasamento foi um fator que colaborou para o sucesso da ideia de Tim. \cite{htmlHist}

\section{A evlução do padrão HTML}
O HTML 2, nomeado assim por Tim Berners-Lee, surgiu para resolver o problema a linguagem que estava se tornando mal-definida conforme diversos navegadores adicionavam novos elementos HTML de uma forma não coordenada. Para tanto Dan Connolly e seus colegas fizeram o trabalho de unir as diferentes variantes e colocar em um documento de rascunho. Feito o rascunho este foi circulado pela Internet para que a comunidade pudesse comentar. O resultado de tal esforço foi a especificação da nova versão do HTML.

A dificuldade em manter o padrão não foi resolvida de fato na versão 2. O HTML 3, que foi publicado como um rascunho em 1995, também sofreu a dificuldade de manter o padrão da linguagem. O rascunho muito longo teve dificuldades em ser ratificado pelo IETF devido ao tamanho do esforço estimado. O problema em manter o padrão era semelhante ao enfrentado antes da especificação, cada navegador implementava diferentes subconjuntos do padrão além de possíveis extensões à linguagem. O HTML 3 não se tornou de fato um padrão.

Por mais que o HTML 3 não se tornou padrão o rascunho fez progressos importantes lidando com tabelas, notas de rodapé, formulários além de incluir o elemento STYLE e o atributo CLASS para encorajar os autores utilizarem estilo em seus documentos. Ainda em 1995 foi apresentado o artigo sobre a internacionalização da Web que previa acabar com a restrição de cunjunto de caracteres para que outros além do Latin-1 pudessem ser utilizados.

A especificação do HTML 3.2 foi endossada em janeiro de 1997 pelo W3C (World Wide Web Consortium) e trazia como um dos novos elementos disponíveis o OBJECT para incorporar objetos, por exemplo applets, dentro de um documento HTML. O HTML 3.2 foi resultado de uma combinação de todas as especificações anteriores e foi amplamente aprovada. \cite{htmlHist}

A versão anterior a especificação 5 é o HTML 4.01, revisão do HTML 4. Entre as diversas melhorias temos mecanismos para folhas de estilo, scripting e quadros. A versão 4 foi desenvolvida com ajuda de experts em internacionalização. A incorporação do RFC2070 realizou a internacionalização do HTML \cite{html4Int}.

\chapter{Tecnologias complementares}

\section{Highcharts}
Highcharts é uma API Javascript utilizada para gerar gráficos em páginas HTML. Os gráficos possuem varias opções de configuração o que os tornam muito flexíveis. A biblioteca Javascript é capaz de gerar diversos tipos de gráficos como: linha, área, coluna, pizza, dispersão, relógio. A API é bastante documentada além de existirem diversos exemplos no site da produto.

\section{PhoneGap}
PhoneGap é um framework para desenvolvimento de aplicações mobile. A aplicação é desenvolvida utilizando HTML5, CSS3 e Javascript e encapsulada na estrutura do framework. Utilizando uma ferramenta de construção a aplicação é empacotada para celulares. O framework é multiplataforma e permite que a apliação seja empacotada para diversos dispositivos como Blackberry, Android, iPhone entre outros. A aplicação acessa os recursos do dispositivo móvel através de bibliotecas Javascript.

\section{Adobe PhoneGap Build}
A ferramenta Adobe PhoneGap Build é um serviço na núvem para a compilação de aplicações que utilizam o framework PhoneGap. A ferramenta permite que o usuário envie o código-fonte da aplicação e em seguida faça o download da mesma já empacotada. A utilização desta ferramenta evita que o usuário tenha que instalar os SDKs nativos de cada plataforma além do Cordova/PhoneGap SDK, necessários para a compilação da aplicação localmente.

\section{Git}
Git é um programa para gerenciamento de código-fonte. Versionamento é uma das principais funcionalidades de tal programa. Uma diferença que pesa a favor do Git em relação a outros programas de gerenciamento de  código-fonte é a possibilidade de saber se arquivos mudam de nome e/ou diretório devido a forma como este funciona. Em outros programas tal informação se deturpa no sentido de que esta é quebrada em duas informações, o arquivo origem é tido como deletado enquanto o arquivo destino é visto como um novo arquivo criado.

\section{GitHub}
GitHub é um serviço de repositório para código-fonte que tem como base o programa de controle de versão Git. Existem tanto versões gratis quanto versões pagas do serviço. No caso de repositórios abertos não existe a necessidade de pagamento. O código-fonte do experimento encontra-se em github.com/camiloperic/xpress.

\section{Queue.js}
