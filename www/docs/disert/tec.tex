\part{Tecnologias}
% ----------------------------------------------------------

% ---
% primeiro capitulo de Resultados
% ---
\chapter{HTML5}
De acordo com o texto de introdução de W3SCHOOLS HTML5 é o padrão HTML mais recente. A versão anterior (4.01) foi lançada em 1999 e como a internet mudou significantemente desde então a versão 5 vem para substituir a versão 4.01, além do XHTML e o HTML DOM Level 2. A nova versão provê de animações a gráficos, musicas a filmes além de possibilitar aplicações web complexas. O padrão HTML5 é multiplataforma e é feito para funcionar em PCs, Tablets, Smartphones e Smart TVs.
	O versão 5 do padrão HTML é uma cooperação entre o World Wide Web Consortium (W3C) e o Web Hypertext Application Tecnology Working Group (WHATWG). Ambas as organizações trabalhavam em diferentes aspectos de aplicações web e em 2006 resolveram se juntar para criar a nova versão do HTML. Algumas diretrizes foram estabelecidas para o futuro empreendimento (W3SCHOOLS):

Novas funcionalidades devem estar baseadas em HTML, CSS, DOM e JavaScript;

A necessidade de plugins externos deve ser reduzida;

Lidar com erros deve ser mais fácil que nas versões anteriores;

“Scripting” deve ser substituido por mais marcações;

O padrão deve ser independente de dispositivo;

O publico deve ter visibilidade do processo de desenvolvimento

Entre as novas “features” disponibilizadas na nova versão temos:

O elemento canvas para desenho em 2D;

Os elementos video e audio para execução de mídias;

Suporte para armazenamento local
